
\section{Optimization Algorithm Description}

\subsection{General algorithm}
Our algorithm does an exhaustive search of spanning trees, while storing the tree ($T$) scoring lowest in the equation

$$
\max \left\{ \sum_{e_i \in T} w(e_i), \sum_{e_i \in T} w(e_{m+1-i}) \right\}
$$

The spanning trees are found by doing so-called cuts and contracts. This creates a recursive computation as follows

If no edge remain, stop the calculation. Otherwise, pick an edge and:

\begin{enumerate}
\item Create a copy of the graph, where the end-points of this edge is merged into one point and any edges between them are removed. Store which edge was contracted. In practise we do this by maintaining a set of edges which should be considered contracted for the current calculations.

\item Create a copy of the graph where the edge is removed, if this disconnects a portion of the graph, stop calculating otherwise calculate all spanning trees on the copy.
\end{enumerate} 

Every leaf of this computation contains a unique spanning tree. It can be found by collecting all stored edges from leaf to root.

\subsection{Tricks}
Since all edges have positive weights, we can maintain the current ``value'' of the calculation that has been created, i.e. the edges that have already been picked. If we maintain a best found solution, we can stop finding more trees once we reach this value, thereby hopefully cutting of branches from the computation tree.

To increase the number of branches we can cut off, we can try to guess a good edge to contract at every step, using one of two heuristics

\begin{enumerate}
\item The edge $e_i$ with the lowest score of $\max\left\{w(e_i),w(e_{m+1-i})\right\}$.

\item The edge $e_i$ with the lowest score of $w(e_i)+w(e_{m+1-i})$
\end{enumerate}


\section{Running time}
If the input graph has $n$ nodes and $m$ edges, we have the following running times.

\begin{itemize}
\item We can calculate all the heuristic values for edges in $O(m)$

\item We can sort the edges by their heuristic value in $O(m)$.

\item For every inner node in the computation tree we do the following:
\begin{itemize}
\item Perform a depth-first-search to see if an edge added to the contracted set, creates a loop. This can be done in $O(n+m)$.

\item Perform a depth-first-search to see if removing an edge splits the current graph in two components. This can be done in $O(m+n)$.
\end{itemize}

\item The computation tree is binary, and every leaf in the tree corresponds to a spanning tree. Cayleys formula states that a complete graph has $O(n^{n-2})$ spanning trees.

\item The path from a leaf to the root of the tree is at most $m$.

\item The computational tree has in the worst case, at most $O(m\cdot n^{n-2})$ nodes.
\end{itemize}

Summing this up, we get a running time of $O((m+n)\cdot m\cdot n^{n-2})$, as we perform two depth-first-searches for every inner node in the computation tree.